\documentclass[oneside,12pt]{report}  

% the dimensions of the page
\textheight=9.25in \topmargin=-0.5in   %See note in Chapter 8 of Sample Report about "Page scaling" option in Adobe
\textwidth=6.0in
\oddsidemargin=0.3in
\evensidemargin=0.3in  % Needed to balance even and odd pages in twoside print copy


% Useful packages
\usepackage{dtklogos}
\usepackage{amsmath}
\usepackage{bm}
%\usepackage[colorlinks=true,pagebackref,linkcolor=blue]{hyperref}
\usepackage{amsfonts}
\usepackage{amsthm}
\usepackage{amsmath}
\usepackage{algorithm}
\usepackage{algorithmic}
\usepackage{graphicx, subfigure}
\usepackage{caption}
\usepackage{excludeonly}

\usepackage{graphicx} 

%\usepackage{doc}
%% Following sets up logic and formatting for conditional twoside copying
%\usepackage{ifthen, color, fancyvrb}
%\usepackage{nextpage}\pagestyle{plain}
%\newcommand\myclearpage{\cleartooddpage
%  [\thispagestyle{empty}]
%  }

\DeclareMathOperator*{\argmin}{arg\ min}
\DeclareMathOperator*{\sign}{sign}

% Note special alternative codes for using TWO bibliographies; see cautionary note in
\DeclareGraphicsExtensions{ps,eps,PNG,png}

% Theorem-like command definitions:
\newtheorem{theorem}{Theorem}[chapter]
\newtheorem{lemma}{Lemma}[chapter]
\newtheorem{definition}{Definition}  % Note, this italicizes everything

% Print the chapter and sections in the toc
\setcounter{tocdepth}{1}

% Specify which files to typeset for this run (note that overall pagination is preserved)
%\includeonly{chapter1, chapter2}
% Specify which files NOT to typeset for this run (note that overall pagination is preserved)
%\excludeonly{}

% Groundwork for allowing double-sided copying with blank versos
\def\prefacesection#1{
\chapter*{#1}
\addcontentsline{toc}{chapter}{#1}
}

\begin{document}


\def\thefootnote{\fnsymbol{footnote}}

\thispagestyle{empty}

% The numbers below controls the amount of space between the following sections
\def\shiftdowna{0.32in}  % Adjust for balance
\def\shiftdownb{0.22in}  % Adjust for balance

% Set up the boiler plate at the top of the page

\begin{center}
\textbf{{\large Mathematical Modeling and Consulting }}\\

\vspace \shiftdowna
\includegraphics[width=0.5\textwidth]{jhu.png}\\

% Home Department
\vspace \shiftdowna
\underline {Sponsor}\\ 
\vspace{5pt}
\textbf{\large The Center for Responsive Politics} \\
\vspace\shiftdowna
\textbf{{Midterm Presentation}}

% TITLE
\vspace \shiftdowna
\textbf{{\Large Measuring Economic Effects of Presidential Elections}}

% STUDENTS
\vspace{0.35in}
\underline {Team}\\
\vspace{5pt}
Yen Theng Tan, Economics Department\\
\texttt{yen@jhu.edu} \\
%\vspace{10pt}
%Jane Doe (Report Coordinator), Home Department
%\texttt{jane.doe@jhu.edu}

% INSTRUCTOR
\vspace \shiftdownb
\underline {Academic Mentor} \\
\vspace{5pt}
\text{Dr.~N.~.H.~Lee}, Applied Mathematics and Statistics\\
\texttt{nhlee@jhu.edu}

% Consultants
%\vspace \shiftdownb 
%\underline {Consultant}\\
%\vspace{5pt}
%Jason Bourne\\

% DATE
\vspace \shiftdowna
Date: Last Complied on \today

\end{center}

\vfill  %Fill page to force following note to bottom
\footnoterule
\noindent \small{This project was supported by the Johns Hopkins University Economics Department.}

% Begin ABSTRACT
\ifthenelse{\boolean{@twoside}}{\myclearpage}{}
\prefacesection{Abstract}

While there exist studies documenting the effects of the economy on a presidential candidate's chances of winning the election, there are few that investigate the effects of the presidential campaign on the economy.  Besides the direct effects from presidential campaigns that are getting progressively more elaborate and more expensive, the increasing political inaction from fear of criticism is indirectly costly as well. We are tasked by our sponsor, the Center for Responsive Politics, to investigate if, during the period before and after historical presidential elections, there exists statiscally significant fluctuations that can be tied to events related to the presidential election, specifically, the announcement of election results. First we identified two measures of the economy - 1) the Standard \& Poor's 500, a stock market index based on the common stock prices of 500 top publicly traded American companies, as determined by S\&P, and 2) the price of U.S. 1-year Treasury Bills (`T-Bills'), short term loans issued to, and backed by the full faith and credit of, the United States Government. These data for these two indicators, generally perceived to be good representations of the market and indicators of the U.S. economy, are collected for the 5 years before and after the 2004 presidential election. The time series of these two indicators are then analyzed using change-point detection. Changepoints are times of discontinuities in a time series that can be induced from changes in observation, and thus a notable occurrence during the announcement of the presidential results would indicate a statistically significant event. From this analysis, we can then come to a reasonable assessment of to what extent the presidential elections affect the economy.


% Begin ACKNOWLEDGMENTS
\ifthenelse{\boolean{@twoside}}{\myclearpage}{}
\prefacesection{Acknowledgments}

I would like to acknowledge Dr. N. H. Lee for his introduction to change-point detection as a statistical tool, and his knowledge of R, without which none of this is possible

% Table of contents, List of Figures, and List of Tables.
\ifthenelse{\boolean{@twoside}}{\myclearpage}{}
\tableofcontents

%\ifthenelse{\boolean{@twoside}}{\myclearpage}{}
%\listoffigures

%\ifthenelse{\boolean{@twoside}}{\myclearpage}{}
%\listoftables


\renewcommand{\thefootnote}{\arabic{footnote}}
\setcounter{footnote}{0}

\ifthenelse{\boolean{@twoside}}{\myclearpage}{}
% Begin Introduction

\ifthenelse{\boolean{@twoside}}{\myclearpage}{}
\prefacesection{Introduction}

The author is Yen Theng Tan. He is currently a senior at Johns Hopkins University, majoring in Applied Mathematics and Statistics, as well as Economics. 



The sponsor is the Center for Responsive Politics, which is the nation's premier research group tracking money in U.S. politics and its effect on elections and public policy. Nonpartisan, independent and nonprofit, the organization aims to create a more educated voter, an involved citizenry and a more transparent and responsive government. In short, the Center's mission is to (1) inform citizens about how money in politics affects their lives, (2) empower voters and activists by providing unbiased information, and (3) advocate for a transparent and responsive government.
%\include{B_TechnicalBackground}
%% Begin Introduction

\ifthenelse{\boolean{@twoside}}{\myclearpage}{}
\prefacesection{Problem Statement}

While there exist studies documenting the effects of the economy on a presidential candidate's chances of winning the election, there are few that investigate the effects of the presidential campaign on the economy.  Presidential campaigns are getting progressively more elaborate and more expensive, with prolific use of the media (and other mediums) to influence and generate voter turnout. While the economy is directly and positively impacted  by such activity, it can also be affected by other indirect factors. For example, in the most recent presidential election, President Barack Obama was caught on camera on assuring outgoing Russian President Dmitry Medvedev that he will have `more flexibility' to deal with contentious issues like missile defense after the presidential election. Such policy deferment practices show that these significant, economy-affecting decisions will be made only after the election is over, thus indicating that the economy was artificially 'held up' during the run-up to the election date.

The sponsor currently has tasked us to investigate if, during the period before and after historical presidential elections, there exists statiscally significant fluctuations that can be tied to events related to the presidential election (i.e. announcement of election results). 

Given that the Center's mission is to (1) inform citizens about how money in politics affects their lives, (2) empower voters and activists by providing unbiased information, and (3) advocate for a transparent and responsive government, 
%\include{D_Analysis}
%\include{E_Results}
%\include{F_Conclusion}

%\include{chapter1}
%\include{chapter2}
%\include{chapter3}
%\include{chapter4}
%\include{chapter5}
%\include{chapter6}


\appendix
\ifthenelse{\boolean{@twoside}}{\myclearpage}{}

\chapter{Lemmas}\label{Lemma}

\chapter{Glossary}\label{Glossary}

\vspace{12pt} 

\vspace{8pt}
\noindent {\bf Ascending node}. The point where the satellite crosses through
the equatorial plane in a northerly direction. 

\vspace{8pt}
\noindent {\bf Earth-centered inertial frame}. A frame of reference whose origin is the center of the earth and which does not rotate with respect to inertial space.

\vspace{8pt}
\noindent {\bf Earth-centered rotating frame}. A frame of reference whose origin is the center of the earth but which rotates with the earth. 

\vspace{8pt} \noindent {\bf Footprint}. The intersection of a visibility cone with the surface of the earth.

\vspace{8pt} \noindent {\bf Great circle of arc}. The shortest path between two points on the surface of the earth. 

\vspace{8pt} \noindent {\bf Groundtrack}.The location of the center of a visibility cone footprint on the surface of the earth.

\vspace{8pt}
\noindent {\bf Inclination}.  The angle between the normal to the orbit plane
and the normal to the equatorial plane.

\vspace{8pt} \noindent {\bf LEO}. An orbit with an altitude approximately below 2,000 km.

\vspace{8pt} \noindent {\bf Molniya orbit}. A highly elliptical orbit with an orbital period of half a day.

\vspace{8pt} \noindent {\bf Projection distance}. The distance between the center of the visibility cone footprint and a point of interest projected onto the plane orthogonal to the vector defining the visibility cone center and tangent to the earth surface.

\vspace{8pt}
\noindent {\bf Right ascension of the ascending node}. The angle
between the unit vector $\bm{X}$ and the point where the satellite crosses the
ascending node, measured counterclockwise when viewed from the north side of
the equatorial plane.


\ifthenelse{\boolean{@twoside}}{\myclearpage}{}
\chapter{Abbreviations}\label{Abbreviations}


\noindent ECI.  Earth-centered inertial frame

\vspace{5pt}

\noindent ECR.  Earth-centered rotating frame

\vspace{5pt}

\noindent LEO.  Low Earth Orbit  

\vspace{5pt}

\noindent RAAN. Right ascension of the ascending node

\vspace{5pt}

%\endinput

% Add your bibliography to Contents
\ifthenelse{\boolean{@twoside}}{\myclearpage}{\newpage}
\addtocontents {toc}{\protect \contentsline {chapter}{REFERENCES}{}}
\addcontentsline{toc}{chapter}{Selected Bibliography Including Cited Works}  

% Bibliography must come last.
\bibliographystyle{plain}
\renewcommand\bibname{Selected Bibliography Including Cited Works}
\nocite{*}  % List ALL references in your references, not just the ones cited in the text.
% This scheme automatically alphabetizes the Bibliography.
\bibliography{Biblio}
\end{document}
