% Begin Introduction

\ifthenelse{\boolean{@twoside}}{\myclearpage}{}
\prefacesection{Problem Statement}

While there exist studies documenting the effects of the economy on a presidential candidate's chances of winning the election, there are few that investigate the effects of the presidential campaign on the economy.  Presidential campaigns are getting progressively more elaborate and more expensive, with prolific use of the media (and other mediums) to influence and generate voter turnout. While the economy is directly and positively impacted  by such activity, it can also be affected by other indirect factors. For example, in the most recent presidential election, President Barack Obama was caught on camera on assuring outgoing Russian President Dmitry Medvedev that he will have `more flexibility' to deal with contentious issues like missile defense after the presidential election. Such policy deferment practices show that these significant, economy-affecting decisions will be made only after the election is over, thus indicating that the economy was artificially 'held up' during the run-up to the election date.

The sponsor currently has tasked us to investigate if, during the period before and after historical presidential elections, there exists statiscally significant fluctuations that can be tied to events related to the presidential election (i.e. announcement of election results). 

Given that the Center's mission is to (1) inform citizens about how money in politics affects their lives, (2) empower voters and activists by providing unbiased information, and (3) advocate for a transparent and responsive government, 